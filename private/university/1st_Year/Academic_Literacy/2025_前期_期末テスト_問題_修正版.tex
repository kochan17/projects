\documentclass[12pt]{article}
\usepackage[utf8]{inputenc}
\usepackage[T1]{fontenc}
\usepackage{amsmath,amssymb,mathtools}
\usepackage{geometry}
\geometry{a4paper,margin=2cm}

\title{演習:問題のみ}
\author{}
\date{}

\begin{document}

\maketitle

\section*{問1(極限を求めよ)}
\begin{enumerate}
  \item $\displaystyle \lim_{x\to0}\,\frac{1}{2}\!\left(\frac{1}{x+3}-\frac{1}{3}\right)$
  \item $\displaystyle \lim_{x\to\infty}\,\frac{2^{x}-3^{x}}{3^{x}+2^{x}}$
  \item $\displaystyle \lim_{x\to0}\,\frac{1-\cos(2x)}{x^{2}}$
\end{enumerate}

\section*{問2(微分せよ)}
\begin{enumerate}
  \item $y=\dfrac{2-3x}{3x+5}$
  \item $y=\log(\log x)$
  \item $y=\sqrt{x}\,\sin x+\dfrac{1}{x^{2}}$
  \item $y=\cos^{-1}(2x)$
  \item $y=\dfrac{\sin x}{1+\cos x}$
\end{enumerate}

\section*{問3(高次導関数)}
$y=(x-3)^{5}$ の $3$ 次導関数を求めよ。

\section*{問4(逆関数の導関数)}
$y=x^{2}-2x\ \ (x>1)$ の逆関数の導関数を求めよ。

\section*{問5(法線)}
曲線 $y=e^{2x}$ の、点 $(0,1)$ を通る法線を求めよ。

\section*{問6(増減と極限)}
$f(x)=\log x$ の増減を調べ、極限を求めよ。

\section*{問7(変曲点)}
$y=2e^{1/x}$ の変曲点を求めよ。

\section*{問8(マクローリン展開)}
$f(x)=\sqrt[4]{\,1+x\,}$ のマクローリン展開を $0$ でない初めの $3$ 項まで求めよ。

\end{document} 