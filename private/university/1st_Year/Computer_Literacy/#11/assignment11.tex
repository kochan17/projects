\documentclass[12pt,a4j]{jarticle}
\usepackage[dvipdfmx]{graphicx}
\usepackage{url}
\begin{document}
\title{コンピュータリテラシレポート#11}
\author{i2524002, 石田幸太}
\date{2025年7月4日}
\maketitle

\section{グループで決めたテーマ、およびグループ全員の名前と学籍番号}

\subsection{テーマ}
「生成AIは現行の民主主義を"理想的な民主主義"へ近づける後押しとなり得るか」

このテーマを選定した理由は、現代民主主義が直面する参加率低下・政治的無関心・複雑化する社会課題への対応という根本的課題に対して、生成AIが技術的解決策を提供できるかという実践的な問題意識があるためである。特に参加型予算制度を具体例として取り上げることで、抽象的な議論に終わらず、実証的な検証が可能となる。

\subsection{グループの名前と学籍番号}

\begin{description}
\item[i2524002] 石田幸太
\item[n2424023] 永井麗音
\item[f2524023] 深堀友翔
\end{description}

\section{討議の準備としてしたこと}

\subsection{石田の準備}
\textbf{立場:「AIは民主参加の量的・質的向上を同時に実現できる」}

行政職での経験から、従来の市民参加制度が時間・人的コストの面で限界に達していることを実感していた。議員対応議会対応や パブリックコメントの集計処理に膨大な労力を要する一方で、実際の参加者は限定的で偏りがある現状に課題意識を持っていた。

調査方法として、デジタル民主主義に詳しい@0xtkgshn氏(渋谷区デジタル民主主義ワークショップ実施者、台湾オードリー・タン氏との講演実績を持つ)に直接話を聞いた。特に台湾vTaiwanやNY The People's Moneyの実証例について詳しく教えていただき、参加率5~8倍向上、合意形成時間30\%短縮という具体的な効果を確認した。

加えて、OECD、People Powered等の国際機関レポートを中心に15件の一次資料を収集し、世界的な動向と日本の自治体事例(世田谷区、品川区、文京区)を比較検討した。

\subsection{深堀の準備}
\textbf{立場:「効率化と熟議の質はトレードオフ関係にある」}

AIによる民主主義の「効率化」に対して、技術決定論の罠への警戒から慎重な立場を取った。特に、AIの抽象型要約がもたらす文脈の喪失と、それが民主的議論の本質を損なう可能性について重点的に調査した。

Computational Social Choice(COMSOC)、Explainable AI(XAI)関連の論文8件を40分かけて精査し、Pol.isやDecidimの技術的仕組みと限界について理解を深めた。また、AIハルシネーション(幻覚)事例を具体的に調査し、民主的合意の「偽装」リスクについて論点を整理した。

反対意見への対応として、「道具 vs 目的」の観点から、AI活用が民主主義の本来の意義(熟議・相互理解・共通善の追求)を見失わせる危険性を論理的に構築した。

\subsection{永井の準備}
\textbf{立場:「段階的導入による現実的な両立を支持」}

行政コスト・監査負担の観点から、AIの全面導入と完全否定の中間的な立場を取った。特に中小自治体での実現可能性を重視し、予算制約・人材不足・技術的リテラシーの現実を踏まえた実装戦略を検討した。

両者の意見を調整する中間案として、「低リスク案件(公園整備等)から開始し、段階的にスケールアップする」アプローチを提案するための資料を準備した。

\section{討議の内容}

グループ討議は、Google Chat上で各自の立場を投稿し、相互にコメントする形式で実施した。以下に主要な発言と議論の流れを要約する。

\begin{description}
\item[石田] 「実証データでは参加率が5~8倍向上し、世田谷区では行政への信頼度も24.8ポイント上昇している\cite{setagaya2023}。効率化と質の向上は両立できる」と主張
\item[深堀] 「その『効率化』が問題だ。AIの抽象型要約は文脈を削ぎ落とし、市民の切実な声を統計的データに還元してしまう。それは民主的議論の本質を損なう\cite{panditharatne2023}」と反論
\item[石田] 「確かに技術偏重は危険だが、@0xtkgshn氏から聞いた台湾vTaiwanの事例では、AIは合意形成の『触媒』として機能し、対立を建設的議論に変えている\cite{cui2024}」
\item[深堀] 「vTaiwanは成功例だが、その成功要因は技術ではなく、『理念とガバナンスが先にある』設計思想だ。AIを導入すれば自動的に民主主義が改善されるわけではない」
\item[永井] 「両者とも正しい。現実的には段階的導入が必要。まず低リスクな案件から始めて、効果測定と改善を重ねることで、技術的信頼性と政治的受容性を両立できる」
\item[石田] 「段階的導入は現実的だ。ただし、『技術よりプロセス設計が重要』という深堀さんの指摘は的確。まず『何をもって公正か』を明示することが前提条件」
\item[深堀] 「そうであれば、Human-in-the-Loop原則とExplainable AI による透明性確保が絶対条件。AIの判断根拠を市民が理解できない限り、民主的正統性は担保されない」
\item[永井] 「3者の合意点として、『段階的導入+プロセス設計重視+透明性確保+継続的効果測定』が現実解として浮上した」
\end{description}

議論を通じて、単純な「AI賛成 vs 反対」ではなく、「どのような条件下で、どのような目的のために、どのような手順でAIを活用するか」という設計論に収束した。

\section{結論}

討議を通じて、私の結論は以下の通りである:

\textbf{「生成AIは、適切な設計と段階的実装により、理想的な民主主義への重要な推進力となり得る」}

\subsection{量的・質的向上の両立可能性}
行政職での経験と@0xtkgshn氏から得た実証例を総合すると、AIは従来不可能だった規模での市民参加を実現しつつ、熟議の質も向上させることができる。台湾vTaiwanでは、Pol.isによる意見クラスタリングが参加者に「より多くの賛同を得られる洗練された意見」を投稿するインセンティブを与え、結果的に合意形成を促進している\cite{tsai2024}。

\subsection{技術決定論の回避}
深堀さんの指摘する「技術 vs 目的」の観点は重要である。AIは民主主義の「手段」であり「目的」ではない。成功の鍵は、まず「理想的な民主主義とは何か」(包摂性・熟議・透明性・説明責任)を定義し、その実現のためにAIを位置づけることである。

\subsection{実装戦略}
永井さんの提案する段階的導入は現実的である。低リスクな案件(公園整備・イベント予算等)から開始し、市民の信頼と技術的ノウハウを蓄積した上で、より重要な政策分野に展開することで、リスクを最小化しつつ効果を最大化できる。

\subsection{最終的な価値判断}
現代民主主義が直面する参加率低下・政治的無関心・複雑化する社会課題に対して、従来手法の限界は明らかである。AIを活用しない選択肢は、事実上、現状維持=民主主義の縮小を意味する。適切な設計により、AIは「多数の声を生かし、公正を守る」デジタル民主主義の加速装置となる\cite{oecd2025}。

重要なのは、リスクを「使わない理由」ではなく「設計課題」として捉え、実装を通じて学習し改善していくことである。これこそが、これからの公共部門と市民社会に求められる姿勢だと確信する。

\section{考察}

今回の課題を通じて、以下の新たな気づきを得た:

\subsection{民主主義の「理想」の多様性}
当初は「理想的な民主主義」を自明のものとして議論を始めたが、討議を通じて、石田(効率性・包摂性重視)、深堀(熟議・質重視)、永井(実現可能性重視)で重視する価値が異なることが明らかになった。AIの活用可能性も、この価値観の違いに依存する部分が大きい。

\subsection{技術と政治の複雑な関係}
AIは「中立的な道具」ではなく、その設計・運用・評価のあらゆる段階で政治的・価値的判断が埋め込まれている。「技術的に可能」と「政治的に望ましい」は別次元の問題であり、後者の議論なしに前者だけを論じることの限界を実感した。

\subsection{実証研究の重要性}
@0xtkgshn氏から聞いた具体的事例は、抽象的な議論に実証的な根拠を与えてくれた。特に、台湾やNYの「成功例」も、その背景にある政治的・社会的条件を理解することで、日本への適用可能性をより冷静に評価できるようになった。

\subsection{グループ討議の価値}
一人で考えているだけでは、自分の立場の前提条件や盲点に気づくことは困難である。異なる観点からの批判・質問・代替案の提示により、より包括的で nuanced な理解に到達できた。これ自体が、民主的議論の価値を体現していると感じる。

\begin{thebibliography}{99}

\bibitem{oecd2025} OECD, \textit{Tackling civic participation challenges with emerging technologies}, OECD Public Governance Reviews, 2025.

\bibitem{setagaya2023} 世田谷区, 『参加型予算効果検証報告書』令和5年度, 世田谷区公式サイト, 2023.

\bibitem{panditharatne2023} M. Panditharatne, D. I. Weiner, D. Kriner, \textit{Artificial Intelligence, Participatory Democracy, and Responsive Government}, Brennan Center for Justice, 2023.

\bibitem{cui2024} P. J. W. Cui, \textit{How Public Participation Can Improve AI Governance: vTaiwan's Initiatives}, Friedrich Naumann Foundation Global Innovation Hub, 2024.

\bibitem{tsai2024} L. L. Tsai, A. Pentland, A. Braley, et al., ``Generative AI for Pro-Democracy Platforms'', MIT Exploration of Generative AI, 2024.

\bibitem{yang2025} J. C. Yang, F. Bachmann, \textit{Bridging Voting and Deliberation with Algorithms: Field Insights from vTaiwan and Kultur Komitee}, FAccT '25, 2025.

\bibitem{faliszewski2023} P. Faliszewski, J. Flis, D. Peters, et al., \textit{Participatory Budgeting: Data, Tools, and Analysis}, Proc. IJCAI-23, pp. 2667–2673, 2023.

\bibitem{boyco2024} M. Boyco, P. Robinson, \textit{Artificial Intelligence: Its potential and ethics in the practice of public participation}, SSRN Working Paper, 2024.

\end{thebibliography}

\section{アンケート}

\subsection{Q1:調べる、討論する、考えるというプロセスはあなたにとってどのように有効でしたか。一人で考えるのと比較して述べてください。}

一人で考える場合、どうしても自分の専門分野(行政実務)や経験に基づいた視点に偏りがちである。今回のグループ討議では、深堀さんの技術的・倫理的観点、永井さんの実装面での現実的配慮により、多角的な検討が可能となった。特に、「効率化と熟議の質はトレードオフか」という根本的な問いは、一人では気づかなかった論点である。また、相手の批判に答えるために論拠を整理し直す過程で、自分の立場もより明確になった。民主的議論の価値を身をもって体験できたことが最大の収穫である。

\subsection{Q2:今回のようなレポートは何がよかったですか。何が大変でしたか。}

良かった点は、抽象的な「民主主義とAI」というテーマを、参加型予算という具体的な制度に絞り込むことで、実証的で建設的な議論ができたことである。また、グループ討議により、単なる文献調査に終わらず、異なる立場間の対話を通じて理解を深められた。

大変だった点は、膨大な先行研究の整理と、短時間でのグループ調整である。特に、AIと民主主義という学際的テーマは、政治学・情報学・行政学・社会学など多分野にまたがるため、基本概念の共有だけでも相当な時間を要した。また、Google Chatでの議論は効率的だが、微妙なニュアンスの伝達には限界があった。

\subsection{Q3:リフレクション(今回の課題で分かったこと)・感想・要望をどうぞ。}

今回の課題で最も印象深かったのは、「技術」と「政治」の関係の複雑さである。AIは中立的な道具ではなく、その設計・運用・評価のあらゆる段階で価値判断が埋め込まれている。したがって、「技術的に可能」な解決策を検討する際にも、「政治的・社会的に望ましい」かどうかを常に問い続ける必要がある。

また、@0xtkgshn氏のような実践者からの直接的な学びの価値を実感した。学術論文だけでは得られない、現場の知見や実装上の工夫について貴重な示唆を得ることができた。

要望として、今後このような学際的テーマを扱う際は、より長期間での調査・議論時間を確保していただけると、さらに深い理解に到達できると思う。また、実際の自治体職員や政策立案者をゲストスピーカーとして招いた講義があれば、理論と実践の架橋がより効果的になると考える。

\end{document} 