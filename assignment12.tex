\documentclass[12pt,a4j]{jarticle}
\usepackage[dvipdfmx]{graphicx}
\usepackage{url}
\begin{document}
\title{コンピュータリテラシレポート#12}
\author{学籍番号: i2524002, 氏名: 石田 幸太}
\date{提出日付: 2025年7月11日}
\maketitle

\section{課題の再掲}

本レポートでは、CL演習ガイド\#12「HTML/CSSによるWebページ記述」の演習3を実施した。

演習3の課題内容は以下の通りである。
\begin{itemize}
\item 自分が作成したWebページに関する説明をおこなうLaTeX文書を作成する
\item Webページには演習1、演習2の小課題の内容、または以下のいずれか1つ以上が含まれること
\begin{itemize}
\item id機能またはclass機能を使って「特定の何かだけ表現が他と違う」ようにする
\item 箇条書のどれかの形を使う
\item 表を使う
\end{itemize}
\end{itemize}

\section{レポートの本文}

\subsection{演習3: Webページの作成}

演習3では、HTMLファイルを実際に作成し、ブラウザでの表示を確認した。特にid機能とclass機能を活用したWebページの作成に取り組んだ。

\subsubsection{作成したWebページの具体的内容}

作成したmypage.htmlファイルの構成は以下の通りである。

\textbf{HTMLの基本構造部分:}
\begin{verbatim}
<!DOCTYPE html>
<html>
<head>
<meta charset="utf-8">
<title>石田のページ</title>
<style type="text/css">
h1 { color: blue; text-decoration: underline; }
p { text-indent: 5mm; background: rgb(180,200,255); }
.highlight { color: red; font-weight: bold; }
#important-section { background: yellow; padding: 10px; }
pre { border: double green 4px; padding: 5px; }
</style>
</head>
\end{verbatim}

\vspace{0.5em}
\noindent\textbf{本文部分の主な要素:}
\begin{itemize}
\item h1見出し「石田です」
\item p要素で自己紹介文
\item class="highlight"を使った強調表現
\item id="important-section"を使った特別なセクション
\item pre要素で俳句を表示
\item ul要素で趣味のリスト
\item table要素で好きな科目の表
\item a要素で大学サイトへのリンク
\end{itemize}

\subsubsection{id機能とclass機能の具体的実装}

演習3の主要課題であるid機能とclass機能を以下のように実装した。

\textbf{class機能の実装:}
重要な単語に「highlight」クラスを適用した。
\begin{verbatim}
<p>今回は<span class="highlight">HTML</span>と
<span class="highlight">CSS</span>について学習しました。</p>
\end{verbatim}

対応するCSSは以下の通りである。
\begin{verbatim}
.highlight { color: red; font-weight: bold; }
\end{verbatim}

\textbf{id機能の実装:}
特別なセクションに「important-section」IDを設定した。
\begin{verbatim}
<div id="important-section">
<h2>重要なお知らせ</h2>
<p>レポート提出期限は7月11日です。</p>
</div>
\end{verbatim}

対応するCSSは以下の通りである。
\begin{verbatim}
#important-section { background: yellow; padding: 10px; }
\end{verbatim}

この実装により、同じ要素種別でも個別に異なるスタイルを適用できることを確認した。

\subsubsection{表の作成}

以下のような表を作成した。
\begin{verbatim}
<table border="2">
<tr><th>科目名</th><th>難易度</th><th>評価</th></tr>
<tr><td>数学</td><td>高</td><td>B</td></tr>
<tr><td>物理</td><td>中</td><td>A</td></tr>
<tr><td>英語</td><td>低</td><td>A</td></tr>
</table>
\end{verbatim}

%(ここに作成したWebページの図を入れる)

\url{file:///Users/kotaishida/projects/assignment12/mypage.html}

\section{考察}

\subsection{構造と表現の分離の実践的理解}

今回の課題を通じて、HTMLとCSSを分離することの実用的な価値を体験できた。

例えば、「highlight」クラスを使って複数の単語を強調した際、後からスタイルを変更したくなった時にCSS側の1行を修正するだけで全ての該当箇所が変更された。これは実際の作業でとても効率的だと感じた。

\subsection{grepツールによる構造化データ抽出の理解}

カリキュラムで説明された「grepなどのツールで大見出しを全部集めて来て一覧を作る」という例について、実際にターミナルで試してみた。

\begin{verbatim}
grep "<h1>" mypage.html
\end{verbatim}

このコマンドで見出しが抽出できることを確認した。もしHTML内に色の指定などが混在していたら、この抽出が複雑になることが理解できた。

\subsection{Webの特殊性の実感}

ブラウザウィンドウのサイズを変更すると、文章が自動的に折り返されることを確認した。これは印刷物とは大きく異なる特徴で、HTMLで構造のみを指定する理由が実感できた。

\section{アンケート}

\subsection{Q1: HTMLによるページの記述はどれくらい知っていましたか。今回やってみてどうでしたか。}

HTMLの基本的な書き方については、以前エンジニアとして働いていた時に使ったことがあるので知っていた。divタグやclassの使い方も実務で覚えていた。

ただし、今回の授業で「なぜHTMLがこのような設計になっているのか」という背景を学術的に学ぶことができた。実務では「こう書けば動く」という感覚で使っていたが、構造と表現の分離という設計思想を理論的に理解できたのは新鮮だった。

\subsection{Q2: CSSによる表現の指定はどれくらい知っていましたか。今回やってみてどうでしたか。}

CSSについても実務で使った経験があるため、基本的な書き方は知っていた。colorやbackgroundなどのプロパティも使ったことがある。

しかし、今回の授業では「なぜCSSとHTMLが分離されているのか」という根本的な理由を学ぶことができた。実務では効率的にコードを書くことに集中していたが、この分離が情報処理の観点から重要な意味を持つことを理解できた。具体的には、HTMLで文書の構造だけを記述しておくことで、プログラムが「見出し」「段落」「表」などの要素を正確に識別できるということだった。もし見た目の指定が混在していると、プログラムによる自動処理が複雑になってしまう。

\subsection{Q3: リフレクション(今回の課題で分かったこと)・感想・要望をどうぞ。}

実務経験はあったが、今回の授業でHTML/CSSの設計思想について学術的に学べたことが最大の収穫だった。

特に印象的だったのは、grepツールでの構造抽出の例だった。実務では「きれいなコードを書く」ことは意識していたが、それがプログラムによる情報抽出を可能にするという観点は新鮮だった。

今後は、単に動くコードを書くだけでなく、なぜそのような設計になっているのかという背景も理解しながら技術を使っていきたいと思う。

\end{document}